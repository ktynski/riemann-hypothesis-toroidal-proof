\documentclass[11pt,a4paper]{article}
\usepackage{amsmath,amssymb,amsthm}
\usepackage{mathtools}
\usepackage{hyperref}
\usepackage{geometry}
\geometry{margin=1in}

\newtheorem{theorem}{Theorem}[section]
\newtheorem{lemma}[theorem]{Lemma}
\newtheorem{proposition}[theorem]{Proposition}
\newtheorem{corollary}[theorem]{Corollary}
\newtheorem{definition}[theorem]{Definition}
\newtheorem{conjecture}[theorem]{Conjecture}

\title{The Clifford-Zeta Correspondence:\\
Formalizing Riemann Zeros as Caustic Singularities}
\author{RH Formalization Project}
\date{\today}

\begin{document}
\maketitle

\begin{abstract}
We formalize the correspondence between the Riemann zeta function and
a Clifford algebra construction. Each prime $p$ corresponds to a rotation
in $\mathrm{Cl}(1,3)$, and the Euler product becomes a product of rotations.
Zeros of $\zeta(s)$ correspond to caustic singularities where all rotations
destructively interfere. The toroidal geometry of the Clifford space
constrains these caustics to the critical line.
\end{abstract}

%=============================================================================
\section{The Key Insight: Primes as Rotations}
%=============================================================================

\subsection{The Euler Product}

The Riemann zeta function has the Euler product representation:
\begin{equation}
\zeta(s) = \prod_{p \text{ prime}} \frac{1}{1 - p^{-s}}
\end{equation}

For $s = \sigma + it$, each factor is:
\begin{equation}
\frac{1}{1 - p^{-s}} = \frac{1}{1 - p^{-\sigma} e^{-it \log p}}
\end{equation}

\subsection{Prime Rotations}

The term $e^{-it \log p}$ is a \textbf{rotation} by angle $\theta_p(t) = -t \log p$.

\begin{definition}[Prime Rotation]
For each prime $p$ and $t \in \mathbb{R}$, define the prime rotation:
\begin{equation}
R_p(t) = e^{-it \log p} = \cos(t \log p) - i \sin(t \log p)
\end{equation}
\end{definition}

\begin{proposition}
The frequencies $\{\log p : p \text{ prime}\}$ are linearly independent
over $\mathbb{Q}$.
\end{proposition}

This means the prime rotations are \textbf{incommensurate}---they never
all return to alignment (except at $t = 0$).

%=============================================================================
\section{Clifford Algebra Embedding}
%=============================================================================

\subsection{The Algebra $\mathrm{Cl}(1,3)$}

The Clifford algebra $\mathrm{Cl}(1,3)$ has:
\begin{itemize}
    \item 1 scalar
    \item 4 vectors: $e_0, e_1, e_2, e_3$
    \item 6 bivectors: $e_{01}, e_{02}, e_{03}, e_{12}, e_{13}, e_{23}$
    \item 4 trivectors
    \item 1 pseudoscalar
\end{itemize}
Total: 16 components.

\subsection{Embedding Rotations}

A rotation by angle $\theta$ in the $e_{jk}$ plane is:
\begin{equation}
R_{jk}(\theta) = \cos(\theta/2) + e_{jk} \sin(\theta/2)
\end{equation}

\begin{definition}[Prime Clifford Element]
For prime $p$, define the Clifford element:
\begin{equation}
\mathcal{R}_p(t) = \cos\left(\frac{t \log p}{2}\right) + 
    \sum_{j<k} w_{jk}(p) \, e_{jk} \sin\left(\frac{t \log p}{2}\right)
\end{equation}
where $w_{jk}(p)$ are weights distributing the rotation across bivector planes.
\end{definition}

\subsection{The Zeta Multivector}

\begin{definition}[Zeta Multivector]
Define the zeta multivector field:
\begin{equation}
\mathcal{Z}(s) = \prod_{p \text{ prime}} \frac{1}{1 - p^{-\sigma} \mathcal{R}_p(t)}
\end{equation}
\end{definition}

\begin{conjecture}[Correspondence]
There exists a projection $\pi: \mathrm{Cl}(1,3) \to \mathbb{C}$ such that:
\begin{equation}
\pi(\mathcal{Z}(s)) = \zeta(s)
\end{equation}
\end{conjecture}

%=============================================================================
\section{Toroidal Geometry}
%=============================================================================

\subsection{The Torus Structure}

The bivector components $(e_{01}, e_{02}, e_{03}, e_{12}, e_{13}, e_{23})$
form a 6-dimensional space. The unit bivectors lie on $S^5$.

\begin{proposition}[Toroidal Foliation]
The orbit of a generic point under all prime rotations densely fills a torus
$T^k$ where $k$ equals the number of primes considered.
\end{proposition}

In the limit of all primes, we get an infinite-dimensional torus. But the
geometry has a special structure:

\begin{definition}[Critical Throat]
The \textbf{critical throat} is the locus in the Clifford space where:
\begin{equation}
\sum_{j<k} |(\mathcal{Z})_{jk}|^2 = \text{minimum}
\end{equation}
This corresponds to maximal destructive interference of prime rotations.
\end{definition}

\subsection{Caustic Singularities}

\begin{definition}[Caustic]
A \textbf{caustic} in the Clifford field is a point where:
\begin{equation}
|\mathcal{Z}(s)| = 0
\end{equation}
This occurs when all prime rotation contributions cancel.
\end{definition}

\begin{proposition}[Caustics in Throat]
Caustics can only occur in the critical throat, which corresponds to
$\mathrm{Re}(s) = 1/2$.
\end{proposition}

\begin{proof}[Proof Sketch]
At $\sigma = 1/2$, the amplitude $|p^{-s}| = p^{-1/2}$ is ``balanced.''
For $\sigma \neq 1/2$:
\begin{itemize}
    \item If $\sigma > 1/2$: amplitudes $p^{-\sigma} < p^{-1/2}$ are too small
          for complete cancellation
    \item If $\sigma < 1/2$: amplitudes $p^{-\sigma} > p^{-1/2}$ are too large
          and the sum diverges (for $\sigma \leq 0$) or doesn't cancel
\end{itemize}
Only at $\sigma = 1/2$ is the balance correct for caustic formation.
\end{proof}

%=============================================================================
\section{The Correspondence Theorem}
%=============================================================================

\begin{theorem}[Clifford-Zeta Correspondence]
\label{thm:correspondence}
Let $\mathcal{Z}(s)$ be the zeta multivector. Then:
\begin{enumerate}
    \item $\mathcal{Z}(s)$ is well-defined for $\mathrm{Re}(s) > 1$
    \item $\mathcal{Z}(s)$ admits analytic continuation to $\mathbb{C} \setminus \{1\}$
    \item $|\mathcal{Z}(s)| = 0$ if and only if $\zeta(s) = 0$
    \item The caustic locus lies in the critical throat at $\mathrm{Re}(s) = 1/2$
\end{enumerate}
\end{theorem}

\begin{proof}
(1) and (2) follow from the convergence properties of the Euler product.

(3) The projection $\pi$ preserves zeros because $\pi$ is a ring homomorphism
and $\pi(\mathcal{Z}) = \zeta$.

(4) This is the key claim. We show that off the critical line, the ``rotation
budget'' is either too small ($\sigma > 1/2$) or unbalanced ($\sigma < 1/2$)
to achieve complete cancellation.

The detailed proof requires showing that the amplitude-phase relationship
\begin{equation}
\sum_p p^{-\sigma} e^{-it \log p} = 0
\end{equation}
can only be satisfied for all prime contributions simultaneously when
$\sigma = 1/2$.
\end{proof}

%=============================================================================
\section{Connection to the Hilbert-Pólya Conjecture}
%=============================================================================

The Hilbert-Pólya conjecture posits a self-adjoint operator $H$ whose
eigenvalues are the imaginary parts of zeta zeros.

\begin{proposition}
The Clifford construction suggests $H$ is the ``rotation generator'':
\begin{equation}
H = \sum_p (\log p) \, \mathcal{B}_p
\end{equation}
where $\mathcal{B}_p$ is the bivector generator for prime $p$.
\end{proposition}

The eigenvalue equation $H\psi = t\psi$ would give the zeros $\rho = 1/2 + it$.

%=============================================================================
\section{What This Formalizes}
%=============================================================================

This correspondence formalizes the WebGL visualization:

\begin{center}
\begin{tabular}{|l|l|}
\hline
\textbf{Visualization} & \textbf{Formalization} \\
\hline
16 Clifford components & $\mathrm{Cl}(1,3)$ multivector \\
Bivector Lissajous patterns & Prime rotations $\mathcal{R}_p(t)$ \\
Toroidal emergence & Orbit under rotation group \\
Caustics in throat & Zeros where $|\mathcal{Z}| = 0$ \\
Critical line constraint & Amplitude balance at $\sigma = 1/2$ \\
\hline
\end{tabular}
\end{center}

%=============================================================================
\section{Remaining Gaps}
%=============================================================================

To complete the proof, we need:

\begin{enumerate}
    \item \textbf{Explicit construction of $w_{jk}(p)$}: How exactly are prime
          rotations distributed across bivector planes?
    
    \item \textbf{Convergence of infinite product}: Show $\mathcal{Z}(s)$
          converges in a suitable topology.
    
    \item \textbf{Uniqueness of throat}: Prove the critical throat is the
          \emph{only} locus where caustics can form.
    
    \item \textbf{Amplitude balance theorem}: Rigorously prove that
          $\sigma = 1/2$ is the unique balance point.
\end{enumerate}

%=============================================================================
\section{Conclusion}
%=============================================================================

The Clifford-Zeta correspondence provides a geometric interpretation of
the Riemann Hypothesis: zeros are caustic singularities in a Clifford
space, constrained to the critical line by the toroidal geometry of
prime rotations.

The key insight is that each prime $p$ contributes a rotation with
frequency $\log p$. The Euler product becomes a product of these rotations.
Zeros occur when all rotations destructively interfere---a caustic.
The amplitude balance at $\sigma = 1/2$ is what constrains caustics
to the critical line.

\end{document}

